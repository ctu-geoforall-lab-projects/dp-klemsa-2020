\chapter{Úvod}
	V životě každého z nás se pravděpodobně někdy přihodilo že jme se zmátli své orientační smysly a ocitli se na jiném místě než jsme čekali. Často můžeme zabloudit pokud nevidíme v dáli žádný pevný bod, například se pohybujeme v mlze. V takovém případě nás lidská mysl dokáže dokonale zmást a i když si myslíme že se pohybujeme po přímce, pokud bysme se koukli na naše stopy shora, budou připomínat spíše zamotaný špagát. Pokud se v takovémto případě dostaneme do potíží, může naše zdraví nebo život záviset na času, ve kterém nás záchranné složky dokáží najít, zejména jedná li se o děti či seniory.
	Pro takovéto případy je důležité mít plán postupu při pátrání po pohřešovaných osobách. Jedná li se o plošné pátrání je jednou z nejpodstatnějších částí speciálně vyhotovený mapový podklad takzvaných pátracích sektorů. Tyto sektory jsou plochy oddělené neprostupnýma bariérama o vhodném tvaru a velikosti, sloužící pátrací četě pro rozdělení velkého prostoru na menší a tak optimalizovat dobu pátrání na minimum.
		
\section{Cíle práce}
	Tato práce si klade za cíl vytvořit rešerši používaných algoritmů pro tvorbu polygonových prvků z množiny linií a na základě poznatků v této oblasti se pokusí navrhnout optimalizovaný algoritmus pro získání polygonů. V další části práce se budeme zabývat implementací algoritmu, který by měl sloužit v praxi pro tvorbu těchto polygonů. Následné optimalizaci pátracích sektorů se v této práci zabývat nebudeme, ta se řeší v diplomové práci \textit{Aplikace pro tvorbu pátracích sektorů na základě přirozených bariér} \cite{sladkova2019aplikace}
	
\section{Obsah práce}

