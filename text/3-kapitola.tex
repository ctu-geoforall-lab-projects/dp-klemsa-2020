\chapter{Vektorové datové modely}
\label{chap:vektorovedatovemodely}
	Každý z nás se setkal s pojmem vektorová grafika. Klasický vektor, tak jak ho z matematiky známe, je charakterizován velikostí a směrem. Ovšem může být také dán dvěma body, čehož v grafických informačních systémech využíváme. Základem každé datové struktury pro GIS je tedy reprezentace bodu. Každý takový bod má dané souřadnice $X$, $Y$, případně $Z$. Třída bodu je pak základní stavební kámen pro další datové typy, jako jsou například linie a polygony. Potřebné atributy k bodům, nebo jiným datovým typům, jsou poté přiřazeny přes identifikátor geometrie. Toto je tedy základní princip datové struktury GIS.
	
\section{Špagetový model}
	Špagetový model je nejjednodušší model který se v GIS používá. Každý prvek, ať se jedná o bod, linii, či polygon, je uložen jako posloupnost souřadnic, v případě bodu má tedy tato posloupnost pouze jeden prvek. Takto velice jednosuše definujeme všechny objekty v ploše, které ovšem nenesou informaci o vztazích mezi sebou. Tyto chybějící vztahy zpomalují většinu prostorových analýz nad daty, kdy při každé operaci musí být potřebné vztahy dopočítány a po dokončení operace jsou vztahy opět ztraceny. Další nevýhoda je například u reprezentace polygonů, kdy společná linie dvou polygonů je v tomto modelu uložena dvakrát, pro každý polygon zvlášť. V tomto modelu se mohou snadno vyskytovat topologicky nekorektní data, kdy se nám například kříží linie a nemají v tomto křížení společný bod, takzvané \textit{fuzzy} průsečíky a mnoho dalších topologických chyb. \cite{QGIS_software} \cite{tucek1997geograficke} \cite{kolar2003geograficke}
	
\section{Topologický model}
	


\section{Hierarchický model}


