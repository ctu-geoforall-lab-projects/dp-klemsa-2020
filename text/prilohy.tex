%%%%%%%%%%%%%%%%%%%%%%%%%%%%%%%%%%%%%%%%%%%%%%%%%%%%%%%%%%%%%%%%%%%%%%%%%%%%%%%%%%%
%%                 PŘÍLOHA - UŽIVATELSKÁ PŘÍRUČKA                                %%
%%%%%%%%%%%%%%%%%%%%%%%%%%%%%%%%%%%%%%%%%%%%%%%%%%%%%%%%%%%%%%%%%%%%%%%%%%%%%%%%%%%
\chapter{Srovnání polygonizace v ArcGIS Desktop a QGIS Desktop}
\label{chap:srovnani}
Srovnání bylo provedeno na testovacích datech, které poskytl \textit{Ing. Jan Růžička, Ph.D.}. Jelikož nemohla být poskytnuta reálná data, jedná se o uměle vygenerované linie, které jsou dle slov doktora Růžičky obdobného charakteru a velikosti. Data obsahují celkem 10000 linií. Tato data byla dále testována ve zmenšené formě a to s 1000 a 100 liniemi.

\section{Parametry počítače}
\begin{itemize}
\item \textbf{Operační systém:} 
\item \textbf{Procesor:} 
\item \textbf{RAM:} 
\end{itemize}

\section{ArcGIS Desktop}

\section{QGIS Desktop}


\chapter{Elektornické přílohy}
\label{user-guide}

\section{CD disk}

\pagenumbering{Roman}
\label{app:cd}
    
    \begin{description}
        \item[\tt BP-DistortionRemover.pdf] ~ \\ text bakalářské práce ve formátu *.pdf,
    	
        
        \item[\tt DistortionRemover/] ~ \\ adresář s programem,
        \begin{description}
        		
        		\item[\tt DistortionRemover.exe] ~ \\ Spustitelný soubor aplikace,
        		
        		\item[\tt License.txt] ~ \\ textový soubor s popisem licence,
        		
            	\item[\tt Source/] ~ \\ adresář obsahující zdrojové kódy,
            	\begin{description}
            			\item[\tt DistortionRemover.pro] ~ \\ projektový soubor,
            			\item[\tt main.cpp] ~ \\ zdrojový kód funkce main,
            			\item[\tt distortionremover.h] ~ \\ hlavičkový soubor distortionremover,
            			\item[\tt distortionremover.cpp] ~ \\ zdrojový soubor distortionremover obsahující funkce pro chod programu,
            			\item[\tt ui\_distortionremover.h] ~ \\ hlavičkový soubor grafického rozhraní,
            			\item[\tt distortionremover.ui] ~ \\ soubor obsahující grafické prostředí v XML,
            			\item[\tt picturetransformation.h] ~ \\ hlavičkový soubor picturetransformation,
            			\item[\tt picturetransformation.cpp] ~ \\ zdrojový soubor picturetransformation obsahující výpočetní funkce,
            			\item[\tt resource.qrc] ~ \\ soubor pro uložení binárních souborů do aplikace který obsahuje ikonu,
            			\item[\tt DR.ico] ~ \\ ikona programu.

            	\end{description}
        \end{description}
    \end{description}

