\chapter{Projekt pátrač} 
\label{chap:searchandrescue}

	Projekt PÁTRAČ vznikl ve spolupráci s policií ČR. Dle elektronické publikace~\cite{chaloupkova2017vyuziti} se jedná o \textit{Využití vyspělých technologií a čichových schopností psů pro zvýšení efektivity vyhledávání pohřešovaných osob v terénu}. Projekt lze rozdělit na dvě základní části. \textbf{Technická část}, která má za cíl vytvořit software pro přípravu a řízení pátracích akcí a \textbf{biologickou část}, která se zabývá sběrem biologických dat v terénu během pátrání u psů/psovodů~\cite{Zeman2009thesis}.
	
	Technická část je dále rozdělena do čtyř samostatných modulů.
\begin{itemize}
	\item Zásuvný modul pro přípravu a řízení operace pro QGIS,
	\item mobilní aplikace pro komunikaci se štábem,
	\item aplikace pro segmentaci prostoru na základě přirozených bariér,
	\item aplikace pro generování datového skladu a projektu pro QGIS.
\end{itemize}	
My se v této práci budeme zabývat jednou z částí přípravy segmentace prostoru na základě přirozených bariér. 

\section{Pátrání po pohřešovaných osobách}
	Podle charakteru prostředí lze záchranné akce rozdělit na několik základních skupin.
	
\subsubsection{Sutinové vyhledávání}
	Jedná se o vyhledávání osob v sutinách budov. V České republice spadá do kompetence hasičských záchranných sborů
	
\subsubsection{Plošné vyhledávání}
	Tento typ pátracích akcí je řízen policií České republiky, především za použití kynologických pátracích týmů. Jedná se o pátrání po jedné či více pohřešovaných osobách na rozsáhlém území. 

\subsubsection{Vyhledávání na vodních plochách}
	Tuto oblast zabezpečují hasičské záchranné sbory.

\subsubsection{Vyhledávání v horském terénu a v lavinách}
	Tuto činnost zabezpečuje Horská služba, která je na práci v obtížném terénu vybavena a vycvičena. 
	
	
\subsection{Základní pojmy}
	Pro lepší pochopení danné problematiky zde budou uvedeny základní pojmy při pátrání po pohřešovaných osobách.
	
\subsubsection{Pohřešovaná osoba}
	Pohřešovaná osoba je osoba, po které bylo vyhlášeno nebo započato pátrání. Tato osoba může být v bezprostředním ohrožení života, proto je nutné osobu v co nejkratším čase vypátrat a provést záchranu. Často se může jednat o osobu která je závislá na cizí péči, dítě, senior, nebo nemocná osoba závislá na zdravotní péči. Život ohrožující můžou být i nepříznivé povětrnostní podmínky, nebo okolní prostředí~\cite{zachrana}.
	
	




	
	
	Využití vyspělých technologií a čichových
schopností psů pro zvýšení efektivity
vyhledávání pohřešovaných osob v terénu






Search and Rescue, v českém jazyce služba pátrání a záchrany. Poskytování této služby je závazné pro všechny členské státy mezinárodní organizace pro civilní letectví. Jedná se o poskytování pomoci lidem, kteří jsou v nouzi následkem letecké nehody. 




Pátrání a záchrana (SAR) je vyhledávání a poskytování pomoci lidem, kteří jsou v nouzi nebo bezprostředním nebezpečí. Obecné pole pátrání a záchrany zahrnuje mnoho speciálních podpolí, obvykle určených typem terénu, ve kterém je pátrání prováděno.

