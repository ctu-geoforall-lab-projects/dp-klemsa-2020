% ML: Pokud se projekt skutecne jmenuje "Patrac", tak by to melo byt s velkym "P"
% TK: Těžko říct 
\chapter{Projekt Pátrač} 
\label{chap:searchandrescue}

Projekt Pátrač vznikl ve spolupráci s Policií ČR. Dle elektronické
publikace~\cite{chaloupkova2017vyuziti} se jedná o \textit{Využití
  vyspělých technologií a čichových schopností psů pro zvýšení
  efektivity vyhledávání pohřešovaných osob v terénu}. Projekt lze
rozdělit na dvě základní části. \textbf{Technická část}, která má za
cíl vytvořit software pro přípravu a řízení pátracích akcí, a
\textbf{biologickou část}, která se zabývá sběrem biologických dat v
terénu během pátrání u psů/psovodů~\cite{Zeman2009thesis}. V této
kapitole bylo čerpáno z publikací~\cite{sladkova2019aplikace,
  Zeman2009thesis, pavlista2009rizeni, zachrana}.
	
	Technická část je dále rozdělena do čtyř samostatných modulů.
\begin{itemize}
	\item Zásuvný modul pro přípravu a řízení operace pro QGIS,
	\item mobilní aplikace pro komunikaci se štábem,
	\item aplikace pro segmentaci prostoru na základě přirozených bariér,
	\item aplikace pro generování datového skladu a projektu pro QGIS.
\end{itemize}	
My se v této práci budeme zabývat jednou z částí přípravy segmentace
prostoru na základě přirozených bariér. Konkrétně se bude jednat o
segmentaci na polygony ze vstupních linií. Tyto polygony jsou pak dále
zpracovávány do vhodných velikostí a tvarů. Následným zpracováním
polygonů se zabývá diplomová práce.

\section{Pátrání po pohřešovaných osobách}
	Podle charakteru prostředí lze záchranné akce rozdělit na několik základních skupin.
	
\begin{itemize}
	\item Sutinové vyhledávání,
	\item plošné vyhledávání,
	\item vyhledávání na vodních plochách,
	\item vyhledávání v horském terénu a v lavinách.
\end{itemize}	
	
Aplikace by měla být využívána především v oblasti plošného
vyhledávání. Jedná se o pátrání po pohřešované osobě či osobách na
rozsáhlém území, které spadá do kompetence Policie České
republiky. Pátrání je prováděno v sektorech, které jsou děleny
přirozenými či uměle vytvořenými bariérami. Vyhledávání v takto
vymezených oblastech může být prováděno za pomoci kynologických skupin
či pátracích rojnic.
	
\subsection{Základní pojmy}
Pro lepší pochopení dané problematiky zde budou vysvětleny základní
pojmy použité v předchozím odstavci.
	
\subsubsection{Pátrání}
Pátrání je činnost, která si klade za cíl nalezení hledaného
objektu. Objektem v tomto případě může být osoba pohřešovaná, s
neznámou totožností či usmrcená. Pátrání však lze vést i za účelem
nalezení movitých věcí, kterými mohou být odcizené dopravní
prostředky, zbraně a podobně.

\subsubsection{Pohřešovaná osoba}
Pohřešovaná osoba je osoba, po které bylo vyhlášeno nebo započato
pátrání. Tato osoba může být v bezprostředním ohrožení života, proto
je nutné osobu v co nej\-krat\-ším čase vypátrat a provést
záchranu. Často se může jednat o osobu, která je závislá na cizí péči,
dítě, seniora, nebo nemocnou osobu závislou na zdravotní péči. Život
ohrožující mohou být i nepříznivé povětrnostní podmínky, nebo okolní
prostředí.
	
\subsubsection{Pátrací sektor}
Je-li známa přibližná lokalita možného výskytu pohřešované osoby,
zpravidla je větší prostor rozdělen na menší, takzvané pátrací
sektory. Sektory jsou ohraničeny ba\-ri\-é\-ra\-mi či jinými hranicemi
a dle členitosti terénu a aktuálních podmínek může být volena velikost
sektorů. Zároveň by měl mít sektor vhodný tvar pro efektivnější pohyb
pátracího týmu.
	
\subsubsection{Bariéra}
Bariérou nazýváme člověkem či přírodně vytvořenou překážkou, kterou
lze jen obtížně překonat a tudíž lze předpokládat, že hledaná osoba
tuto bariéru nepřekonala. Kromě bariér dělí pátrací sektory další
linie, které mohou napomoct při postupu pátrání a orientaci pátracích
týmů v terénu. Přesná skladba těchto linií nám však není známa,
jelikož se jedná o citlivá data.
