\chapter{Závěr}
\label{9-zaver}

		V této práci byla zpracována stručná rešerše literatury, zabývající se metodami odstranění zkreslení z fotografických snímků. Vybrané metody odstranění zkreslení jsou dále v práci popsány. Jako nejvhodnější způsob se jevilo odstranění zkreslení za pomoci \textit{Brownova} distorzního modelu, který je v obdobných softwarech často používaný.
	
	Hlavním cílem této práce bylo vytvořit softwarové řešení tohoto problému, které bude mít moderní grafické rozhraní. Pro tvorbu softwaru byl zvolen programovací jazyk \textit{C++} s využitím knihoven \textit{Qt}. Tento nástroj usnadnil tvorbu grafického rozhraní, které je zkoncipováno tak, aby mohl uživatel intuitivně program ovládat a nebylo zapotřebí podrobnější znalosti softwaru. Program byl sestaven pro platformu \textit{Windows}, jelikož je však jazyk \textit{C++} s knihovnami \textit{Qt} určen pro různé platformy, nic nebrání tento program sestavit i pro \textit{Linux} či \textit{macOS}, což tento software dělá použitelnější pro větší okruh lidí. Dalším vylepšením vytvořeného programu \textit{DistortionRemover} by do budoucna mohlo být obohacení o výpočet parametrů prvků vnitřní orientace kamery, které v současné době musí uživatel získat z jiného softwaru, či jiného zdroje. Velkým zlepšením by mohlo také být práce s obrazovými daty typu RAW, zamezilo by se tak velkým ztrátám kvality při opravě snímku. Samozřejmostí je opravení chyb, v programu, které budou odhaleny rozsáhlejším používáním. 
		
	Ve finální části, je předvedena práce s programem na příkladu historického stavebního objektu z databáze PhotoPa a jsou nastíněny další možné postupy zpracování.

	

