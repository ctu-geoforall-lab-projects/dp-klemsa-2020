\chapter{Rešerše používaných algoritmů}
\label{chap:reserzepouzivanychalgoritmu}
	Tato kapitola se zabývá přehledem využívaných algoritmů pro tvorbu polygonů z množiny linií. Polygonizace, tak jak jí můžeme pozorovat v různých GIS softwarech, se zpravidla neřeší jedním algoritmem, ale používají se různé algoritmy pro jednotlivé kroky polygonizace. Popíšeme zde tedy do jakých kroků lze polygonizaci rozdělit a jak jednotlivé kroky optimálně řešit.

	Polygonizaci lze nejjednodušeji rozdělit na 2 základní kroky. Jako vstupní data budeme uvažovat množinu linií, tyto linie ovšem nemusí obsahovat, v reálném použití často neobsahují, body ve společných průsečících. Tato situace nastává především použiváme-li různé vrstvy pro tvorbu polygonů. V terminologii GIS se tyto průsečíky označují jako \textit{fuzzy}, volně přeloženo jako \textit{neostré}. Abychom mohli zahájit další krok polygonizace, je nutné tyto průsečíky doplnit. V první části se tedy budeme zabývat, jak optimálně doplnit průsečíky množin linií.
	
	Předpokládejme že množina linií je zbavená \textit{fuzzy} průsečíků, tedy každá dvojice úseček má společný nanejvýš jeden koncový bod. V takovéto situaci můžeme přistoupit k dalšímu kroku, tedy vlastní tvorbě polygonů z množiny linií. Polygony, které hledáme, musí splňovat podmínku, že žádný námi vytvořený polygon nemůže být rozdělen jakoukoliv linií na více menších polygonů. Jelikož se v podstatě jedná o určité vyjádření grafu, pro tvorbu polygonů se využívá grafových algoritmů.
	
\begin{figure}[h]
    \centering % <-- added
\begin{subfigure}{0.5\textwidth}
  \includegraphics[width=\linewidth]{./pictures/5/fuzzy_1.pdf}
  \caption{Množina linií s fuzzy průsečíky}
  \label{fig:2-fuzzy_1}
\end{subfigure}\hfil % <-- added
\begin{subfigure}{0.5\textwidth}
  \includegraphics[width=\linewidth]{./pictures/5/fuzzy_2.pdf}
  \caption{Množina linií doplněná o průsečíky}
  \label{fig:2-fuzzy_2}
\end{subfigure}\hfil % <-- added
\begin{subfigure}{0.5\textwidth}
  \includegraphics[width=\linewidth]{./pictures/5/fuzzy_3.pdf}
  \caption{Vytvořené polygony}
  \label{fig:2-fuzzy_2}
\end{subfigure}\hfil % <-- added
\caption{Znázornění postupu polygonizace}
\end{figure}

\section{Výpočet průsečíků množiny linií}
Výpočet všech průsečíků množin linií lze provést snadno, testováním všech úseček se všemi. Tímto postupem ovšem zjevně dosáhneme složitosti $\mathcal{O}(N^2)$ kde $N$ je počet segmentů linie. Pro výpočet průsečíků lze ovšem využít i algoritmus s časovou náročnosti $\mathcal{O}(n\log{}n)$, známý také jako Bentley–Ottmannův algoritmus.

\subsection{Vzájemná poloha dvou úseček}
Ve 2D výpočetní geometrii jsou standardně jednotlivé segmenty linie vyjádřeny počátečními a koncovými souřadnicemi. Uvažujme tedy že máme dány 2 přímky $p_1 = |S_1 E_1|$ a $p_2 = |S_2 E_2|$, kde $S_1 = [x_{S1},y_{S1}]$, $E_1 = [x_{E1},y_{E1}]$, $S_2 = [x_{S2},y_{S2}]$, $E_2 = [x_{E2},y_{E2}]$ a potřebujeme provést test, zda se dané přímky protínají či nikoliv, můžeme použít takzvaný \textit{Half-Plane} test, tedy test, který určuje zda bod leží v pravé či levé polorovině od přímky. Tento test zopakujeme celkem čtyřikrát a to na počáteční a koncový bod druhé úsečky, abysme zjistili zda se přímky protínají. Test je založen na výpočtu orientace dvou vektorů $\vec{u}$  a $\vec{v}$, kde vektor $\vec{u}$ je směrový vektor úsečky, tedy $\overrightarrow{S_1E_1}$ a vektor $\vec{v}$ je vektor $\overrightarrow{S_1S_2}$. Po aplikaci výpočtu orientace vektorů na každý z koncových bodů jsme schopni jednoduše určit zda se dané přímky protínají. Tento postup nám ovšem nedá souřadnice samotného průsečíku.

\begin{figure}[h]
  \centering
  \includegraphics[width=7.5cm]{./pictures/5/half-plane_vector.pdf}
  \caption{\textit{Half-plane} test}
  \label{fig:2-half_plane_vector}
\end{figure}

\begin{figure}[h]
    \centering % <-- added
\begin{subfigure}{0.5\textwidth}
  \includegraphics[width=\linewidth]{./pictures/5/half-plane_1.pdf}
  \caption{Test bodu $S_2$ k přímce $|S_1E_1|$}
  \label{fig:2-half_plane_1}
\end{subfigure}\hfil % <-- added
\begin{subfigure}{0.5\textwidth}
  \includegraphics[width=\linewidth]{./pictures/5/half-plane_2.pdf}
  \caption{Test bodu $E_2$ k přímce $|S_1E_1|$}
  \label{fig:2-half_plane_2}
\end{subfigure}\hfil % <-- added

\medskip
\begin{subfigure}{0.5\textwidth}
  \includegraphics[width=\linewidth]{./pictures/5/half-plane_3.pdf}
  \caption{Test bodu $S_1$ k přímce $|S_2E_2|$}
  \label{fig:2-half_plane_3}
\end{subfigure}\hfil % <-- added
\begin{subfigure}{0.5\textwidth}
  \includegraphics[width=\linewidth]{./pictures/5/half-plane_4.pdf}
  \caption{Test bodu $E_1$ k přímce $|S_2E_2|$}
  \label{fig:2-half_plane_4}
\end{subfigure}\hfil % <-- added
\caption{Grafické znázornění \textit{Half-plane} testů pro zjištění existence průsečíku}
\label{fig:2-half_plane}
\end{figure}

\subsection{Nalezení průsečíku dvou úseček}
	Pro nalezení průsečíku $P$ dvou úseček vycházíme z parametrického vyjádření přímek, jelikož obecné rovnice neumožňují pracovat s přímkami rovnoběžnými s osou y. Uvažujme tedy stejné značení bodů jako v předchozím odstavci. Nejprve je tedy nutné vyjádřit rovnici přímky z počátečního a koncového vrcholu. Pro parametrické vyjádření využijeme směrový vektor přímky a libovolný bod na přímce. Ovšem pro zjednodušení výpočtů použijeme pro přímku $p_1$ směrový vektor $\overrightarrow{S_1E_1}$ a jako bod do parametrické rovnice dosadíme $S_1$, obdobně zkonstruujeme i parametrické vyjádření přímky $p_2$. Soustava rovnic pro výpočet průsečíku $p_i$ pak tedy bude vypadat takto:
	
\begin{align*} 
P = S_1 + s \cdot \overrightarrow{S_1E_1}, \\
P = S_2 + t \cdot \overrightarrow{S_2E_2}. \\
\end{align*}

Rozepsáno podle souřadnic:

\begin{align*} 
x_P = x_{S1} + s(x_{E1} - x_{S1}), \\
y_P = y_{S1} + s(y_{E1} - y_{S1}), \\
x_P = x_{S2} + t(x_{E2} - x_{S2}), \\
y_P = y_{S2} + t(y_{E2} - y_{S2}).
\end{align*}

Pro získání průsečíku musíme nejprve ze soustavy rovnic vyjádřit parametry $s$ a $t$. To provedeme následujícím způsobem:

\begin{align*} 
s = \frac{y_{S1}(x_{S2} - x_{E2}) + y_{S2}(x_{E2} - x_{S1}) + y_{E2}(x_{S1} - x_{S2})}{(x_{E1} - x_{S1}) (y_{S2} - y_{E2})   -   (y_{E1} - y_{S1}) (x_{S2} - x_{E2})}, \\
t = \frac{y_{S1}(x_{S2} - x_{E1}) + y_{E1}(x_{S1} - x_{S2}) + y_{S2}(x_{E1} - x_{S1})}{(x_{E1} - x_{S1}) (y_{S2} - y_{E2})   -   (y_{E1} - y_{S1}) (x_{S2} - x_{E2})}.
\end{align*}


Z charakteru parametrického vyjádření přímek poté můžeme rozhodnout o vzájemné poloze úseček. Pokud parametry $s \in <0,1>, t \in <0,1>$ pak se úsečky protínají, v opačném případě se úsečky neprotínají. Je potřeba ošetřit případy kdy úsečky leží na jedné přímce. Podrobněji je tento způsob popsán v publikaci~\cite{bayer2008algoritmy}.


\subsection{Bentley–Ottmannův algoritmus}
Tento algoritmus je založen na technice zvané \textit{sweep line} neboli \textit{zametací přímka}. Tato technika je již zmíněna v kapitole \ref{chap:navrhovanialgoritmu}. Hlavní myšlenka pro zrychlení algoritmu oproti metodě hrubé síly je počítat průsečíky pouze sousedních linií, tedy linií které jsou právě protnuty zametací přímkou. Tímto způsobem se vyhneme časové složitosti $\mathcal{O}(N^2)$ a docílíme mnohem lepšího výsledku, tedy složitosti $\mathcal{O}(n\log{}n)$~\cite{bentley1979algorithms}.

Prvním a zásadním krokem algoritmů využívající techniku \textit{zametací přímky} je setřídit vstupní data podle jedné ze souřadnic. Setřídíme tedy jednotlivé souřadnice podle jedné z nich, často se používá souřadnice $x$, pak můžeme vizualizovat \textit{zametací přímku} jako svislou linii pohybující se zleva doprava. Toto setřízení lze provést jedním ze známých třídících algoritmů v čase $\mathcal{O}(n\log{}n)$.

	Setříděné souřadnice jsou pak vloženy do fronty, odkud jsou postupně zpracovávány a hledány průsečíky jen těch linií, které jsou aktuálně protnuty zametací přímkou, dokud algoritmus nenalezne veškeré průsečíky. Podrobnější popis algoritmu najdeme v publikacích~\cite{coupland, bentley1979algorithms}.
	
\begin{figure}[h]
  \centering
  \includegraphics[width=10cm]{./pictures/5/Bentley-Ottmann-plane-sweep-algorithm.png}
  \caption{Znázornění postupu zametací přímky v Bentley-Ottmanově algoritmu. Převzato z~\citep{coupland}.}
  \label{fig:5-bentley_ottman}
\end{figure}


% https://www.cs.princeton.edu/~rs/AlgsDS07/17GeometricSearch.pdf
% https://courses.csail.mit.edu/6.006/spring11/lectures/lec24.pdf
% https://cw.fel.cvut.cz/old/_media/courses/ae4m39vg/lectures/09-intersect.pdf

\section{Tvorba polygonů z množiny linií}
Předpokládejme že máme množinu linií doplněnou o průsečíky, tedy každá dvojice úseček v této množině sdílí nejvýše koncový bod. V předchozí sekci bylo řečeno že průsečíky linií lze doplnit pomocí \textit{Bentley–Ottmannova} algoritmu v čase $\mathcal{O}(n\log{}n)$. Nyní chceme tedy v této množině linií nalézt všechny polygony, tak aby jednotlivé polygony spolu sdílely maximálně hrany a zároveň aby žádný z polygonů neobsahoval jiný polygon. Takové linie jsou v podstatě reprezentací grafu. K nalezení polygonů můžeme použít tedy grafových algoritmů. 

	Tak jak je polygonizace popsána v publikaci \cite{joaquim2003polygon}, je prvním krokem k nalezení polygonů, \textbf{nalezení nejkratších cest v grafu}. Pro nalezení nejkratších cest nám může posloužit jeden ze známých algoritmů. Nalezení nejkratších cest mezi všemi dvojicemi vrcholů v grafu jsme schopni realizovat v čase $\mathcal{O}(n^3)$, kde $n$ je počet vrcholů. Můžeme využít například 
	
\begin{itemize}
\item Floydův–Warshallův algoritmus,
\item Dijkstrův algoritmus,
\item Johnsonův algoritmus.
\end{itemize}
	
	Jako druhý krok lze považovat \textbf{nalezení nejkratších cyklů}, tedy vpodstatě vlastních polygonů. V publikaci \cite{joaquim2003polygon} je pro nalezení nejkratších cyklů použitý jednoduchý hladový algoritmus.
