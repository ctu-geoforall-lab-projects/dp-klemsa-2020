\chapter{Implementace algoritmu do GeoTools}
\label{chap:implementacealgoritmudogeotools}
	Pro implementaci polygonizačního algoritmu byla vybrána open-source knihovna pro programovací jazyk java \textit{GeoTools}. V současné době je \textit{GeoTools} součástí \textit{OSGeo}, což je nezisková organizace, snažící se podporovat open-source geoprostorové technologie. Mimo \textit{GeoTools} organizace zastupuje řadu dalších projektů, mezi které patří v desktopových aplikacích textit{QGIS}, \textit{GRASS GIS}, mezi mapové servery \textit{MapServer}, \textit{GeoServer} a mezi knihovnami jsou nejvýznamějšími zástupci \textit{GEOS}, \textit{GDAL}, \textit{PROJ}, nebo \textit{JTS}. Právě zmíněná knihovna \textit{JTS} tvoří základní kámen pro \textit{GeoTools}, jelikož jsou v ní definovány základní geometrické tipy a operace pro práci s prostorovými daty. Možná více známá knihovna než \textit{JTS} je knihovna \textit{GEOS}, což není nic jiného než přepis knihovny z javy do C++.
	
	
\
	
	
	jako je například \textit{QGIS}, \textit{GRASS GIS}, \textit{MapServer},

	

